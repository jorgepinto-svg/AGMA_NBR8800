
    
\documentclass[11pt]{article}
\usepackage{amsfonts,amssymb,amsmath}
\usepackage{float}
\usepackage[portuguese]{babel}
\usepackage{geometry}
\usepackage{titlesec}
\usepackage{enumitem}

\geometry{
    paperwidth=210mm,        % Largura da página (A4)
    paperheight=297mm,       % Altura da página (A4)
    left=2cm,                % Margem esquerda
    right=2cm,               % Margem direita
    top=.1cm,                 % Margem superior
    bottom=1cm,              % Margem inferior
    heightrounded,           % Ajusta a altura da página para ser um número inteiro de linhas
    includehead,             % Inclui o cabeçalho no cálculo da altura total da página
    includefoot              % Inclui o rodapé no cálculo da altura total da página
}

\setcounter{secnumdepth}{5}

\titleformat{\paragraph}
{\normalfont\normalsize\bfseries}{\theparagraph}{1em}{}
\titlespacing*{\paragraph}{0pt}{3.25ex plus 1ex minus .2ex}{1.5ex plus .2ex}

\titleformat{\subparagraph}
{\normalfont\normalsize\bfseries}{\thesubparagraph}{1em}{}
\titlespacing*{\subparagraph}{0pt}{3.25ex plus 1ex minus .2ex}{1.5ex plus .2ex}




\begin{document}

\tableofcontents % Gera o índice automaticamente

\section{Introdução}
Esta memória de cálculo tem como objetivo demonstrar a verificação completa do perfil metálico chamado \textit{PM-006}, a qual foi 
adotada a seção \textit{W200X52}, localizado a $0m$ a partir do ponto inicial da barra, seguindo as premissas estabelecidas na norma NBR8800:2024 
(Projeto de estruturas de aço e de estruturas mistas de aço e concreto de edificações).
\\



\section{Documentos de Referência}
Os documentos de referência utilizados para elaboração desta memória são:\\
NBR8800:2024 - Projeto de estruturas de aço e de estruturas mistas de aço e concreto de edificações;\\
NBR6120:2019 - Ações para o cálculo de estruturas de edificações;\\
NBR6123:2023 - Forças devidas ao vento em edificações.\\





\section{Propriedades dos Perfil}
\subsection{Propriedades Geométricas}
As dimensões do perfil são apresentadas nesta seção a seguir:
\begin{table}[H]
\def\arraystretch{2}
\caption{Dimensões do perfil - propriedades geométricas da seção}
\begin{center}
\begin{tabular}{|c||c|c|c|c|c|c|}
    \hline
        $Barra$   &   $Perfil$    &  $h (mm)$   &  $b_f (mm)$   &   $t_f (mm)$   &  $t_w (mm)$   &   $A (cm^2)$  \\ \hline            
        PM-006   & W200X52 &   206.0   &   204.0     &    12.6     &    7.9    &    66.5  \\ \hline 
\end{tabular}
\end{center}
\end{table}

Sendo:\\
$h$ é a altura do perfil;\\
$b_f$ é a largura do perfil;\\
$t_f$ é a espessura da mesa;\\
$t_w$ é a espessura da alma;\\
$A$ é a área do perfil.
\\



\subsection{Propriedades Mecânicas}
As caracterísitcas geométricas do perfil estão disponíveis na tabelas a seguir:

\begin{table}[H]
\def\arraystretch{1.3}
\caption{Dimensões do perfil - propriedades geométricas da seção (parte 1/2)}
\begin{center}
\begin{tabular}{|c||c|c|c|c|c|c|}
\hline
Barra   &   Perfil      &  $J (cm^4)$       &  $I_x (cm^4)$  &  $I_y (cm^4)$ & $A_{w,y} (cm^2)$  & $A_{w,x} (cm^2)$     \\ \hline            
PM-006 & W200X52 &   32.0  &     5300.0   &   1800.0    &   16.2        &   42.8           \\ \hline 
\end{tabular}
\end{center}
\end{table}

\begin{table}[H]
\def\arraystretch{1.3}
\caption{Dimensões do perfil - propriedades geométricas da seção (parte 2/2)}
\begin{center}
\begin{tabular}{|c||c|c|c|c|c|c|c|}
\hline
Barra   &    Perfil     &    $W_x (cm^3)$    &  $W_y (cm^3)$    &   $Z_x (cm^3)$    &    $Z_y (cm^3)$   &   $r_x (mm)$   &   $r_y (mm)$    \\ \hline            
PM-006 & W200X52 &      514.0      &     174.0     &     569.0      &      264.0     &    51.591    &     89.19    \\ \hline 
\end{tabular}
\end{center}
\end{table}

Sendo:\\
$J$ é a constante de torção;\\
$I_x$ é o momento de inércia no eixo de maior inércia;\\
$I_y$ é o momento de inércia no eixo de menor Inércia;\\
$A_{w,y}$ é a área efetiva de cisalhamento na direção vertical;\\
$A_{w,x}$ é a área efetiva de cisalhamento na direção horizontal;\\
$W_x$ é o módulo de resistência elástico no eixo de maior inércia;\\
$W_y$ é o módulo de resistência elástico no eixo de menor inércia;\\
$Z_x$ é o módulo de resistência plástico no eixo de maior inércia;\\
$Z_y$ é o módulo de resistência plástico no eixo de menor inércia;\\
$r_x$ é o raio de giração em torno do eixo de maior inércia;\\
$r_y$ é o raio de giração em torno do eixo de menor inércia.\\



O raio de giração polar da seção bruta em relação ao centro de cisalhamento, calculado conforme a seguinte equação: 

$$r_0 =\sqrt{{{r_x}^2 + {r_y}^2}} = \sqrt{{0.0516^2} + {0.0892^2}} = 103.0mm$$
\\

Constante de empenamento da seção transversal, calculado conforme item D.2.8.a) da NBR8800:2024

$$C_w = \frac{I_y(d-t_f)^2}{4} = \frac{1800.0*10^{-8} * (0.206-0.0126)^2}{4} = 1.68316*10^{-7}m^6$$
\\



\section{Esforços Solicitantes do Perfil}
Para determinação dos esforços internos nas barras da estrutura, foi utilizado o software SAP2000 v.23.3.1, 
para modelagem, lançamento de carregamento e análise da estrutura. O software SAP2000 é um programa de análise
estrutural desenvolvido pela Computers and Structures, Inc. (CSI), que é amplamente utilizado para modelagem, 
análise e dimensionamento de estruturas com foco em engenharia civil (concreto e metálica). 

Para esta análise foi escolhido o ponto localizado a $0m$ do ponto inicial da barra \textit{PM-006}. Ao avaliar
todas as combinações de esforços que esta barra esta submetida, a combinação \textit{ELU 001} é a que impõe os maiores 
esforços internos sobre a seção. A seguir são apresentando os esforços internos que foram considerado na verificação do perfil:

\begin{table}[H]
    \def\arraystretch{1.3}
    \caption{Esforçco Solicitantes Internos na Barra}
    \begin{center}
    \begin{tabular}{|p{1cm}||p{2.5cm}|p{1.1cm}|p{1.1cm}|p{1.1cm}|p{1.1cm}|p{1.1cm}|p{1.1cm}|}
        \hline
            Barra   & Combinação      & P  $(kN)$   &   V2 $(kN)$  &   V3 $(kN)$  &    T $(kN.m)$ &	   M2 $(kN.m)$   &     M3 $(kN.m)$   \\ \hline            
            PM-006 & ELU 001    &      -295.18    &      0    &      0    &       0     &       0       &       0        \\ \hline 
    \end{tabular}
    \end{center}
    \end{table}

Sendo: \\
$P$ é a Força Axial na mesma direção do eixo do perfil;\\
$V2$ é a Força Cortante na mesma direção da alma do perfil;\\
$V3$ é a Força Cortante na mesma direção das mesas do perfil;\\
$T$ é o Momentor Torsor na mesma direção do eixo do perfil;\\
$M2$ é o Momento Fletor que atua no eixo de maior inércia;\\ 
$M3$ é o Momento Flexor que atua no eixo de menor inércia.\\




\section{Esforços Resistentes do Perfil}
Tendo em vista que o esforço axial solicitante do perfil, no ponto escolhido para análise, é um esforço de compressão ou tração, 
então a verificação adotada é aquela descrita no item 5.3 da Norma NBR8800:2024. 



\subsection{Força Axial de Compressão Resistente}
Para barra comprimidas, deve ser atendida a condição:

$$N_{c,Sd} \leq N_{c,Rd}$$

Onde:\\
$N_{c,Sd}$ é a força axial de compressão solicitante de cálculo;\\
$N_{c,Rd}$ é a força axial de compressão resistente de cálculo, determinada conforme 5.3.2.\\

Para se determinar o $N_{c,Rd}$, antes é necessário saber qual é o valor da força axial de flambagem ($N_e$), que será visto no próximo item desta memória.\\



\subsubsection{Força Axial de Flambagem}
A força axial de flambagem de uma barra com seção transversal duplamente simétrica ou simétrica em relação a um ponto é o menor dos três valores dados a seguir:\\

a) para flambagem por flexão em relação ao eixo central de inércia x da seção transversal:

$$N_{ex} = \frac{{\pi ^2}.E.I_x}{{L_x}^2} = \frac{3.1415^2 * (199.948*10^9) * (5300.0*10^{-8}) }{({1*3})^2} = 11620.5 kN$$
\\

b) para flambagem por flexão em relação ao eixo central de inércia y da seção transversal:

$$N_{ey} = \frac{{\pi ^2}.E.I_y}{{L_y}^2} = \frac{3.1415^2 * (199.948*10^9) * (1800.0*10^{-8}) }{({1*3})^2} = 3946.6 kN$$
\\


c) para flambagem por torção em relação ao eixo longitudinal z (que passa pelo centro de cisalhamento):

$$N_{ez} = \frac{1}{r_0^2}{\left[\frac{{\pi ^2}.E.C_w}{L_z^2} + G.J \right]} = $$
$$\frac{1}{(103.0*10^{-3})^2} {\left[\frac{{3.1415 ^2} * (199.948*10^9) * (1.68316*10^{-7})}{({1*3})^2} + (76.9031*10^6) * (32.0*10^{-8}) \right]} = 5794.1 kN$$
\\

Onde:

$L_x$ é o comprimento destravado associado à flexão em relação ao eixo $x$;

$I_x$ é o momento de inércia da seção transversal em relação ao eixo $x$;

$L_y$ é o comprimento destravado associado à flexão em relação ao eixo $y$;

$I_y$ é o momento de inércia da seção transversal em relação ao eixo $y$;

$L_z$ é o comprimento destravado associado à torção;

$E$ é o módulo de elasticidade do aço;

$C_w$ é a constante de empenamento da seção transversal;

$G$ é o módulo de elasticidade transversal do aço;

$J$ é a constante de torção da seção transversal;

$r_0$ é o raio de giração polar da seção bruta em relação ao centro de cisalhamento, conforme previamente detalhado. 
\\

Como o menor valor entre $N_{ex}$, $N_{ey}$ e $N_{ez}$ é $3946.6$, então $N_e = 3946.6 kN$.\\

Com o valor de $N_e$ determinado, deve-se voltar para o item 5.3.2 da NBR8800:2024 para determinar o Índice de
esbeltez reduzido ($\lambda_0$), Conforme item  em 5.3.3.2.

 

\subsubsection{Índice de Esbeltez Reduzido ($\lambda_0$) }

O índice de esbeltez reduzido ($\lambda_0$) é calculado pela seguinte equação:

$$\lambda_0 = \sqrt{\frac{A_g.f_y}{N_e}} = \sqrt{\frac{0.0066 * (344.7379*10^6) }{3946.6*10^3 }} = 0.762$$

Com $\lambda_0$ determinado, deve-se calcular o valor do fator de redução associado à resistência à compressão ($\chi$).\\



\subsubsection{Fator de Redução Associado à Resistência à Compressão ($\chi$)}

O fator de redução associado à resistência à compressão ($\chi$) é calculado pela seguinte equação:\\

\( \rightarrow  \) para $\lambda_0 \leq 1.5$ : 
$$\chi = 0.658^{{\lambda_0}^2}$$ 
\\

\( \rightarrow  \) para $\lambda_0 \textgreater 1.5$ : 
$$\chi=\frac{0.877}{{\lambda_0}^2}$$
\\



Como ${\lambda_0} \leq 1.5$, então:
$$\chi = 0.658^{{\lambda_0}^2}=0.658^{0.762^2}=0.7842$$
\\

   
Com $\chi$ determinado, o último valor a ser determinado é área efetiva da seção transversal da barra ($A_{ef}$), conforme item 5.3.4 da NBR8800:2024.\\



\subsubsection{Área Efetiva da Seção Transversal da Barra ($A_{ef}$)}
Os elementos que fazem parte das seções transversais usuais para efeito de flambagem local, 
são classificados em:

\( \rightarrow  \) (AA) duas bordas longitudinais apoiadas como, por exemplo, almas de seções I, H ou U; e

\( \rightarrow  \) (AL) uma borda longitudinal apoiada e outra livre como, por exemplo, mesas de seções I, H, T ou U laminadas.
\\

A área efetiva da seção transversal ($A_{ef}$) deve ser considerada igual à área bruta ($A_g$) se 
todos os elementos componentes da seção transversal (AA ou AL) possuírem razão entre largura
e espessura ($b/t$) igual ou inferior ao valor $((b/t)_{lim}/\sqrt{\chi})$.
\\

Vale ressaltar que:

\( \rightarrow  \) para elementos do tipo AA, ($b/t$) é a razão entre distância entre mesas do perfil e espessura da alma ($(h-2.t_f)/t_w$);

\( \rightarrow  \) para elementos do tipo AL, ($b/t$) é a razão entre meia largura do perfil e espessura da mesa ($0.5b/t_f$).
\\

Como a \textbf{largura efetiva da alma} é em função da esbeltez da chapa que compõem a alma, deve-se verificar
dentro das faixas em que este valor está trabalhando. A partir disto, aplicar uma das duas expressões a seguir para 
o valor de \textbf{largura efetiva da alma}:\\
                                     
\( \rightarrow  \) para $\displaystyle \frac{(h-2.t_f)}{t_w} \leq \frac{(b/t)_{lim}}{\sqrt{\chi}}$: $b_{ef,alma} = (h-2.t_f)$ ou 
\\[10pt]

\( \rightarrow  \) para $\displaystyle \frac{(h-2.t_f)}{t_w} \textgreater \frac{(b/t)_{lim}}{\sqrt{\chi}}$: $\displaystyle  b_{ef,alma}=(h-2.t_f)\left(1-0.18{\sqrt{\frac{\sigma_{el}}{\chi f_y}}}\right) \sqrt{\frac{\sigma_{el}}{\chi f_y}}$; 
\\[10pt]

{sendo $\displaystyle \sigma_{el} = \left(1.31{\frac{{(b/t)_{lim}}}{(h-2.t_f)/t_w}}\right)^2 f_y$}
\\[25pt]



O valor limite para esbeltez da alma é:\\

$$(b/t)_{lim} = 1.49\sqrt{\frac{E}{f_y}} = 1.49\sqrt{\frac{ 199.948*10^9 }{ 344.7379*10^6 }} = 35.9$$

\vspace{0.3cm}

A relação entre esbeltez da alma e o valor limite é:\\
\\
\centerline {$\displaystyle \frac{(h-2.t_f)}{t_w} \leq \frac{(b/t)_{lim}}{\sqrt{\chi}}$ \( \Rightarrow  \) 
$\displaystyle \frac{(0.206-2*0.0126)}{0.0079} \leq \frac{35.9}{\sqrt{0.7842}}$}

\vspace{0.3cm}

Como a esbeltez da alma é menor ou igual do que o valor limite, a \textbf{largura efetiva da alma} é dada por:\\

$$b_{ef,alma} = (h-2.t_f) = (0.206 - 2*0.0126) = 0.1808m$$

\vspace{0.3cm}


   

Como a \textbf{largura efetiva da mesa} é em função da esbeltez de meia chapa que compõem a mesa, deve-se verificar
dentro das faixas em que este valor está trabalhando. A partir disto, aplicar uma das duas expressões a seguir para 
o valor de \textbf{largura efetiva da mesa}:\\
                                     
\( \rightarrow  \) para $\displaystyle \frac{0.5b}{t_f} \leq \frac{(b/t)_{lim}}{\sqrt{\chi}}$: $b_{ef} = 0.5b$ ou
\\[10pt]
                                     
 \( \rightarrow  \) para $\displaystyle \frac{0.5b}{t_f} \textgreater \frac{(b/t)_{lim}}{\sqrt{\chi}}$: $\displaystyle  b_{ef}=0.5b\left(1-0.22{\sqrt{\frac{\sigma_{el}}{\chi f_y}}}\right) \sqrt{\frac{\sigma_{el}}{\chi f_y}}$; 
\\[10pt]

{sendo $\displaystyle \sigma_{el} = \left(1.49{\frac{{(b/t)_{lim}}}{(0.5b)/t_f}}\right)^2 f_y$}
\\[25pt]



 

   

 

   

O valor limite para esbeltez da mesa é:\\

$$(b/t)_{lim} = 0.56\sqrt{\frac{E}{f_y}} = 0.56\sqrt{\frac{199.948*10^9 }{344.7379*10^6 }} = 13.5$$

\vspace{0.3cm}


A relação entre esbeltez da mesa e o valor limite é:\\

\centerline {$\displaystyle \frac{(0.5b)}{t_f} \leq \frac{(b/t)_{lim}}{\sqrt{\chi}}$ \( \Rightarrow  \) 
$\displaystyle \frac{(0.5*0.204)}{0.0126} \leq \frac{13.5}{\sqrt{0.7842}}$}

\vspace{.3cm}

Como a esbeltez da mesa é menor ou igual do que o valor limite, a \textbf{largura efetiva da mesa} é dada por:\\
$$b_{ef,mesa} = (0.5b) = (0.5*0.204) = 0.102m$$

\vspace{0.3cm}

   

Com as larguras efetivas da alma e das mesas, basta determinar a Área Efetiva da Seção Transversal da Barra ($A_{ef}$) como sendo o somatório da multiplicação 
das larguras efetiva pela as respectivas espessuras. Sendo assim:

$$A_{ef} = b_{ef,alma}.t_w + 4.(b_{ef,mesa}.t_f) = 0.1808*0.0079 + 4*(0.102*0.0126) = 65.6*10^{-4} m^2$$
\\



\subsubsection{Força Axial de Compressão Resistente}
Ao final, com todos os parâmetros necessários definidos, basta aplicar a equação do item 5.3.2 da NBR8800:2024. 
A força axial de compressão resistente de cálculo de uma barra, associada aos estados-limite
últimos de instabilidade por flexão, por torção ou flexo-torção, e de instabilidade local, deve ser
determinada pela equação:

$$N_{c,Rd} = \frac{\chi A_{ef} f_y}{\gamma_{a1}} = \frac{0.7842*(65.6*10^{-4})*(344.7379*10^6) }{1.1} = 1613.1kN $$
\\[15pt]

   

ATENÇÃO:\\
Deve ser feita uma verificação mais detalhada caso o ponto selecionado seja uma região de ligação.
\\
\\
\\

                                                                                                                                                                  

\section{Resultados}
\subsection{Verificação Esforço Axial Simples}
Quando o perfil verificado não está simultaneamente sujeito a flexão nos dois eixos ortogonais, a verificação deve atender a seguinte expressão:\\
\\
$N_{Sd}/N_{Rd} \leq 1.0$:\\
\\
Onde:\\
$N_{Sd}$ é a força axial solicitante de cálculo;\\
$N_{Rd}$ é a força axial resistente de cálculo;\\

\\

Como $N_{Sd} = -295.18$ e $N_{Rd} = 1613.1$, então $N_{Sd}/N_{Rd} = 0.183$.\\

                              

\section{Conclusão}
O perfil metálico chamado \textit{PM-006}, a qual foi adotada a seção \textit{W200X52}, e na posição 0m encontra-se:\\
\\

\begin{enumerate}[label=\alph*), leftmargin=1.5cm]


                                        
\item APTO a resistir os esforços axiais simples, conforme detalhado no item intitulado "Verificação Esforço Axial Simples";\\

                     

\end{enumerate}

\end{document}

